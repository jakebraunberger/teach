\documentclass[11pt,article,oneside]{memoir}
\usepackage[left=2cm, right=2cm, top=2cm]{geometry}

\setsecheadstyle{\Large\scshape}
\setsubsecheadstyle{\large\scshape\hspace{5mm}}

\begin{document}

\title{\LARGE Geometry}
\author{\Large Mr. Braunberger \newline \\ \footnotesize\texttt{\noindent jake.braunberger@gmail.com}}
\date{\hspace*{2.3cm}Fall 2018 \newline \noindent \hspace*{-1.5cm} 12:16pm--1:06pm}

\setlength{\epigraphwidth}{.8\textwidth}

\maketitle

\epigraph{I think it is very important--at least it was to me--that if you are going to teach people to make observations, you should show that something wonderful can come from them.}{\textit{--Richard Feynman}}




\section*{Course Description}
Geometry is a branch of mathematics concerned with questions of shape, size, relative position
of figures, and the properties of a space. Geometry has applications in many fields, including art, architecture, 
physics, and math.\footnote{Wikipedia} Upon successful completion of this course, the student will, among other things, have
developed critical thinking skills and have an appreciation for how geometry is useful in everyday life.


\section*{Procedures, Materials, and Expectations}
\subsection*{Materials}
	Everyday, the student is expected to bring to class their issued textbook, a pencil or pen, and a notebook. The student is responsible
for the textbook issued to them and any damage it incurs.
\subsection*{Expectations}
	Math is a subject that builds upon itself and therefore requires study on a daily basis.
	Assignments will be issued daily and are to be completed on time. Late work will be accepted; however,
late assignments will be automatically docked 30\%. Assignments may be completed in pen or pencil, as long as the assignment is neat.

The course will model mathematical thought: errors are to be expected, discovered, and remedied. Any incorrect answers
on homework will be awarded 50\% credit if recompleted correctly within a week of receiving the homework.
\subsection*{Disciplinary Procedure}
	I've heard good things about this particular class section, and I don't forsee any major issues. Issues will be handled as they arise.
\subsection*{Grading}
	Tests and quizzes will be worth 15\% of the grade. The final will be worth 20\% of the grade. Homework will be worth 65\% of the grade.




\section*{The Book}
We will be using \textit{Core Connections Geometry} by Dietiker and Kassarjian to supplement the course. I have no expectation
that the student will read the book, but we will reference it for some homeworks, and the student is 
encouraged to use it as a learning resource.



\section*{Course Outline}
\subsection*{Introduction to Geometry}
\begin{enumerate}
\item Identify, name, and classify figures.
\item Measurement: Distance and midpoint formulas.
\item Perimeter and area.
\end{enumerate}
\subsection*{Reasoning}
\begin{enumerate}
\item Compare and contrast inductive and deductive reasoning.
\item Conditional statements with emphasis on the converse statement.
\item Two-column proofs.
\end{enumerate}
\subsection*{Parallel and Perpendicular Lines}
\begin{enumerate}
\item Reasoning skills on parallel and perpendicular lines.
\item Identify angles formed by such lines.
\item Prove theorems about such lines.
\end{enumerate}
\subsection*{Congruent Triangles}
\begin{enumerate}
\item Classify triangles.
\item Prove triangles congruent.
\item Use theorems about isosceles and equilateral triangles.
\end{enumerate}
\subsection*{Relationships within Triangles}
\begin{enumerate}
\item Write a coordinate proof.
\item Use relationships within triangles: perpendicular bisectors, angle bisectors, medians, and altitudes.
\end{enumerate}



\end{document}