\documentclass[12pt]{article}

\usepackage[margin=1in]{geometry} 
\usepackage{amsmath,amsthm,amssymb}
\usepackage{float}
\usepackage[pdftex]{graphicx}
\usepackage{sagetex}
\usepackage{ifthen}

 

\newcommand{\N}{\mathbb{N}}
\newcommand{\Z}{\mathbb{Z}}

 

\newenvironment{theorem}[2][Theorem]{\begin{trivlist}
\item[\hskip \labelsep {\bfseries #1}\hskip \labelsep {\bfseries #2.}]}{\end{trivlist}}

\newenvironment{lemma}[2][Lemma]{\begin{trivlist}
\item[\hskip \labelsep {\bfseries #1}\hskip \labelsep {\bfseries #2.}]}{\end{trivlist}}

\newenvironment{exercise}[2][Exercise]{\begin{trivlist}
\item[\hskip \labelsep {\bfseries #1}\hskip \labelsep {\bfseries #2.}]}{\end{trivlist}}

\newenvironment{problem}[2][Problem]{\begin{trivlist}
\item[\hskip \labelsep {\bfseries #1}\hskip \labelsep {\bfseries #2.}]}{\end{trivlist}}

\newenvironment{question}[2][Question]{\begin{trivlist}
\item[\hskip \labelsep {\bfseries #1}\hskip \labelsep {\bfseries #2.}]}{\end{trivlist}}

\newenvironment{corollary}[2][Corollary]{\begin{trivlist}
\item[\hskip \labelsep {\bfseries #1}\hskip \labelsep {\bfseries #2.}]}{\end{trivlist}}

 \newenvironment{example}[2][Example]{\begin{trivlist}
\item[\hskip \labelsep {\bfseries #1}\hskip \labelsep {\bfseries #2.}]}{\end{trivlist}}
















\begin{document}
\title{Homework 3}
\author{xxx}
\maketitle



\begin{sagesilent}
from random import randint
import random
from datetime import datetime
random.seed(datetime.now())
import numpy as np
var('A,B,C,D')
\end{sagesilent}




%ppps
\begin{sagesilent}
A = randint(10,100)
B = randint(10,200)
C = randint(10,300)
D = B
P = line([[A,B],[C,D]])
Q = point([A,B])
S = point([C,D])
\end{sagesilent}

\sageplot{P+Q+S}

\begin{problem}{1}
Given $A(\sage{A},\,\sage{B})$ and $B(\sage{C},\,\sage{D})$,
calculate the midpoint of $\overline{AB}$ and the distance
$\mid \overline{AB}\mid $.
\end{problem}
%pppe


%ppps
\begin{sagesilent}
A = randint(10,100)
B = randint(10,200)
C = A
D = randint(10,50)
P = line([[A,B],[C,D]])
Q = point([A,B])
S = point([C,D])
\end{sagesilent}

\sageplot{P+Q+S}

\begin{problem}{2}
Given $A(\sage{A},\,\sage{B})$ and $B(\sage{C},\,\sage{D})$,
calculate the midpoint of $\overline{AB}$ and the distance
$\mid \overline{AB}\mid $.
\end{problem}
%pppe


%ppps
\begin{sagesilent}
A = randint(10,100)
B = randint(10,200)
C = randint(10,300)
D = randint(10,50)
P = line([[A,B],[C,D]])
Q = point([A,B])
S = point([C,D])
\end{sagesilent}

\sageplot{P+Q+S}

\begin{problem}{3}
Given $A(\sage{A},\,\sage{B})$ and $B(\sage{C},\,\sage{D})$,
calculate the midpoint of $\overline{AB}$.
\end{problem}
%pppe


%ppps
\begin{sagesilent}
var('m')
y = function('y')
m = randint(-5,5)
A = randint(10,100)
B = randint(10,200)
C = randint(10,300)
D = randint(10,50)
P = line([[A,B],[C,D]])
Q = point([A,B])
LA = text('A', (A,B+.1*max(D,B,max(y(A),y(C)))))
S = point([C,D])
LS = text('B', (C,D))
y(x) = (B+D)/2.0 + m*(x-(A+C)/2.0)
LA = text('A', (A,B+.03*max(D,B,max(y(A),y(C)))))
LS = text('B', (C,D+.03*max(D,B,max(y(A),y(C)))))
T = plot(y, (x, min(A,C), max(A,C)))
\end{sagesilent}

\sageplot{P+Q+S+T+LA+LS}

\begin{problem}{4}
Given $A(\sage{A},\,\sage{B})$ and $B(\sage{C},\,\sage{D})$,
calculate the midpoint of $\overline{AB}$.
Then, determine if
\[
y(x) = \sage{y(x)}
\]
bisects $\overline{AB}$.
\end{problem}
%pppe



%ppps
\begin{sagesilent}
var('m')
y = function('y')
m = randint(-5,5)
A = randint(10,100)
B = randint(10,200)
C = randint(10,300)
D = randint(10,50)
P = line([[A,B],[C,D]])
Q = point([A,B])
LA = text('A', (A,B+.03*max(D,B,max(y(A),y(C)))))
S = point([C,D])
LS = text('B', (C,D))
y(x) = (B+D)/2.0 + m*(x-(A+C)/2) + 1
LA = text('A', (A,B+.03*max(D,B,max(y(A),y(C)))))
LS = text('B', (C,D+.03*max(D,B,max(y(A),y(C)))))
T = plot(y, (x, min(A,C), max(A,C)))
\end{sagesilent}

\sageplot{P+Q+S+T+LA+LS}

\begin{problem}{5}
Given $A(\sage{A},\,\sage{B})$ and $B(\sage{C},\,\sage{D})$,
calculate the midpoint of $\overline{AB}$.
Then, determine if
\[
y(x) = \sage{y(x)}
\]
bisects $\overline{AB}$.
\end{problem}
%pppe


%ppps
\begin{sagesilent}
B = randint(4,20)
C = randint(10, 1000)
\end{sagesilent}



\end{document}