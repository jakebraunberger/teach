%hhhs

\documentclass[12pt]{article}

\usepackage[margin=1in]{geometry} 
\usepackage{amsmath,amsthm,amssymb}
\usepackage{float}
\usepackage[pdftex]{graphicx}
\usepackage{sagetex}
\usepackage{ifthen}
\usepackage{sagetex}

 

\newcommand{\N}{\mathbb{N}}
\newcommand{\Z}{\mathbb{Z}}

 

\newenvironment{theorem}[2][Theorem]{\begin{trivlist}
\item[\hskip \labelsep {\bfseries #1}\hskip \labelsep {\bfseries #2.}]}{\end{trivlist}}

\newenvironment{lemma}[2][Lemma]{\begin{trivlist}
\item[\hskip \labelsep {\bfseries #1}\hskip \labelsep {\bfseries #2.}]}{\end{trivlist}}

\newenvironment{exercise}[2][Exercise]{\begin{trivlist}
\item[\hskip \labelsep {\bfseries #1}\hskip \labelsep {\bfseries #2.}]}{\end{trivlist}}

\newenvironment{problem}[2][Problem]{\begin{trivlist}
\item[\hskip \labelsep {\bfseries #1}\hskip \labelsep {\bfseries #2.}]}{\end{trivlist}}

\newenvironment{question}[2][Question]{\begin{trivlist}
\item[\hskip \labelsep {\bfseries #1}\hskip \labelsep {\bfseries #2.}]}{\end{trivlist}}

\newenvironment{corollary}[2][Corollary]{\begin{trivlist}
\item[\hskip \labelsep {\bfseries #1}\hskip \labelsep {\bfseries #2.}]}{\end{trivlist}}

 \newenvironment{example}[2][Example]{\begin{trivlist}
\item[\hskip \labelsep {\bfseries #1}\hskip \labelsep {\bfseries #2.}]}{\end{trivlist}}
















\begin{document}
\title{Homework 3}
\author{test}
\maketitle



\begin{sagesilent}
from random import randint
import random
from datetime import datetime
random.seed(datetime.now())
import numpy as np
var('A,B,C,D')
\end{sagesilent}



%vertical line

%ppps5

\begin{sagesilent}
A = randint(10,100)
B = randint(10,200)
C = A
D = randint(10,50)
P = line([[A,B],[C,D]])
Q = point([A,B])
S = point([C,D])
\end{sagesilent}

\sageplot{P+Q+S}

\begin{problem}{2}
Given $A(\sage{A},\,\sage{B})$ and $B(\sage{C},\,\sage{D})$,
calculate the distance
$\mid \overline{AB}\mid $.
\end{problem}
%pppe







% both horizontal and vertical

%ppps5

\begin{sagesilent}
var('m')
y = function('y')
m = randint(-5,5)
A = randint(10,100)
B = randint(10,200)
C = randint(10,300)
D = B
E = C
F = randint(10, 300)

P = line([[A,B],[C,D]])
Q = line([[C,D],[E,F]])
R = point([A,B])
S = point([C,D])
T = point([E,F])


LA = text('A', (A,B+.03*max(D,B,max(y(A),y(C)))))
LB = text('B', (C+.03*C, D))
LC = text('C', (1.03*E, F))

\end{sagesilent}

\sageplot{P+Q+R+S+T+LA+LB+LC}

\begin{problem}{5}
Given $A(\sage{A},\,\sage{B})$ and $B(\sage{C},\,\sage{D})$,
calculate the distance $\mid \overline{AB} \mid$.
\end{problem}

%pppe
%hhhs

\documentclass[12pt]{article}

\usepackage[margin=1in]{geometry} 
\usepackage{amsmath,amsthm,amssymb}
\usepackage{float}
\usepackage[pdftex]{graphicx}
\usepackage{sagetex}
\usepackage{ifthen}
\usepackage{sagetex}

 

\newcommand{\N}{\mathbb{N}}
\newcommand{\Z}{\mathbb{Z}}

 

\newenvironment{theorem}[2][Theorem]{\begin{trivlist}
\item[\hskip \labelsep {\bfseries #1}\hskip \labelsep {\bfseries #2.}]}{\end{trivlist}}

\newenvironment{lemma}[2][Lemma]{\begin{trivlist}
\item[\hskip \labelsep {\bfseries #1}\hskip \labelsep {\bfseries #2.}]}{\end{trivlist}}

\newenvironment{exercise}[2][Exercise]{\begin{trivlist}
\item[\hskip \labelsep {\bfseries #1}\hskip \labelsep {\bfseries #2.}]}{\end{trivlist}}

\newenvironment{problem}[2][Problem]{\begin{trivlist}
\item[\hskip \labelsep {\bfseries #1}\hskip \labelsep {\bfseries #2.}]}{\end{trivlist}}

\newenvironment{question}[2][Question]{\begin{trivlist}
\item[\hskip \labelsep {\bfseries #1}\hskip \labelsep {\bfseries #2.}]}{\end{trivlist}}

\newenvironment{corollary}[2][Corollary]{\begin{trivlist}
\item[\hskip \labelsep {\bfseries #1}\hskip \labelsep {\bfseries #2.}]}{\end{trivlist}}

 \newenvironment{example}[2][Example]{\begin{trivlist}
\item[\hskip \labelsep {\bfseries #1}\hskip \labelsep {\bfseries #2.}]}{\end{trivlist}}
















\begin{document}
\title{Homework 3}
\author{test}
\maketitle



\begin{sagesilent}
from random import randint
import random
from datetime import datetime
random.seed(datetime.now())
import numpy as np
var('A,B,C,D')
\end{sagesilent}

%hhhe


%horizontal line

%ppps5

\begin{sagesilent}
A = randint(10,100)
B = randint(10,200)
C = randint(10,300)
D = B
P = line([[A,B],[C,D]])
Q = point([A,B])
S = point([C,D])
\end{sagesilent}

\sageplot{P+Q+S}

\begin{problem}{1}
Given $A(\sage{A},\,\sage{B})$ and $B(\sage{C},\,\sage{D})$,
$\mid \overline{AB}\mid $.
\end{problem}
%pppe
%hhhs

\documentclass[12pt]{article}

\usepackage[margin=1in]{geometry} 
\usepackage{amsmath,amsthm,amssymb}
\usepackage{float}
\usepackage[pdftex]{graphicx}
\usepackage{sagetex}
\usepackage{ifthen}
\usepackage{sagetex}

 

\newcommand{\N}{\mathbb{N}}
\newcommand{\Z}{\mathbb{Z}}

 

\newenvironment{theorem}[2][Theorem]{\begin{trivlist}
\item[\hskip \labelsep {\bfseries #1}\hskip \labelsep {\bfseries #2.}]}{\end{trivlist}}

\newenvironment{lemma}[2][Lemma]{\begin{trivlist}
\item[\hskip \labelsep {\bfseries #1}\hskip \labelsep {\bfseries #2.}]}{\end{trivlist}}

\newenvironment{exercise}[2][Exercise]{\begin{trivlist}
\item[\hskip \labelsep {\bfseries #1}\hskip \labelsep {\bfseries #2.}]}{\end{trivlist}}

\newenvironment{problem}[2][Problem]{\begin{trivlist}
\item[\hskip \labelsep {\bfseries #1}\hskip \labelsep {\bfseries #2.}]}{\end{trivlist}}

\newenvironment{question}[2][Question]{\begin{trivlist}
\item[\hskip \labelsep {\bfseries #1}\hskip \labelsep {\bfseries #2.}]}{\end{trivlist}}

\newenvironment{corollary}[2][Corollary]{\begin{trivlist}
\item[\hskip \labelsep {\bfseries #1}\hskip \labelsep {\bfseries #2.}]}{\end{trivlist}}

 \newenvironment{example}[2][Example]{\begin{trivlist}
\item[\hskip \labelsep {\bfseries #1}\hskip \labelsep {\bfseries #2.}]}{\end{trivlist}}
















\begin{document}
\title{Homework 3}
\author{test}
\maketitle



\begin{sagesilent}
from random import randint
import random
from datetime import datetime
random.seed(datetime.now())
import numpy as np
var('A,B,C,D')
\end{sagesilent}

%hhhe


%horizontal line

%ppps5

\begin{sagesilent}
A = randint(10,100)
B = randint(10,200)
C = randint(10,300)
D = B
P = line([[A,B],[C,D]])
Q = point([A,B])
S = point([C,D])
\end{sagesilent}

\sageplot{P+Q+S}

\begin{problem}{1}
Given $A(\sage{A},\,\sage{B})$ and $B(\sage{C},\,\sage{D})$,
$\mid \overline{AB}\mid $.
\end{problem}
%pppe







% both horizontal and vertical

%ppps5

\begin{sagesilent}
var('m')
y = function('y')
m = randint(-5,5)
A = randint(10,100)
B = randint(10,200)
C = randint(10,300)
D = B
E = C
F = randint(10, 300)

P = line([[A,B],[C,D]])
Q = line([[C,D],[E,F]])
R = point([A,B])
S = point([C,D])
T = point([E,F])


LA = text('A', (A,B+.03*max(D,B,max(y(A),y(C)))))
LB = text('B', (C+.03*C, D))
LC = text('C', (1.03*E, F))

\end{sagesilent}

\sageplot{P+Q+R+S+T+LA+LB+LC}

\begin{problem}{5}
Given $A(\sage{A},\,\sage{B})$ and $B(\sage{C},\,\sage{D})$,
calculate the distance $\mid \overline{AB} \mid$.
\end{problem}

%pppe







% both horizontal and vertical

%ppps5

\begin{sagesilent}
var('m')
y = function('y')
m = randint(-5,5)
A = randint(10,100)
B = randint(10,200)
C = randint(10,300)
D = B
E = C
F = randint(10, 300)

P = line([[A,B],[C,D]])
Q = line([[C,D],[E,F]])
R = point([A,B])
S = point([C,D])
T = point([E,F])


LA = text('A', (A,B+.03*max(D,B,max(y(A),y(C)))))
LB = text('B', (C+.03*C, D))
LC = text('C', (1.03*E, F))

\end{sagesilent}

\sageplot{P+Q+R+S+T+LA+LB+LC}

\begin{problem}{5}
Given $A(\sage{A},\,\sage{B})$ and $B(\sage{C},\,\sage{D})$,
calculate the distance $\mid \overline{AB} \mid$.
\end{problem}

%pppe
%hhhs

\documentclass[12pt]{article}

\usepackage[margin=1in]{geometry} 
\usepackage{amsmath,amsthm,amssymb}
\usepackage{float}
\usepackage[pdftex]{graphicx}
\usepackage{sagetex}
\usepackage{ifthen}
\usepackage{sagetex}

 

\newcommand{\N}{\mathbb{N}}
\newcommand{\Z}{\mathbb{Z}}

 

\newenvironment{theorem}[2][Theorem]{\begin{trivlist}
\item[\hskip \labelsep {\bfseries #1}\hskip \labelsep {\bfseries #2.}]}{\end{trivlist}}

\newenvironment{lemma}[2][Lemma]{\begin{trivlist}
\item[\hskip \labelsep {\bfseries #1}\hskip \labelsep {\bfseries #2.}]}{\end{trivlist}}

\newenvironment{exercise}[2][Exercise]{\begin{trivlist}
\item[\hskip \labelsep {\bfseries #1}\hskip \labelsep {\bfseries #2.}]}{\end{trivlist}}

\newenvironment{problem}[2][Problem]{\begin{trivlist}
\item[\hskip \labelsep {\bfseries #1}\hskip \labelsep {\bfseries #2.}]}{\end{trivlist}}

\newenvironment{question}[2][Question]{\begin{trivlist}
\item[\hskip \labelsep {\bfseries #1}\hskip \labelsep {\bfseries #2.}]}{\end{trivlist}}

\newenvironment{corollary}[2][Corollary]{\begin{trivlist}
\item[\hskip \labelsep {\bfseries #1}\hskip \labelsep {\bfseries #2.}]}{\end{trivlist}}

 \newenvironment{example}[2][Example]{\begin{trivlist}
\item[\hskip \labelsep {\bfseries #1}\hskip \labelsep {\bfseries #2.}]}{\end{trivlist}}
















\begin{document}
\title{Homework 3}
\author{test}
\maketitle



\begin{sagesilent}
from random import randint
import random
from datetime import datetime
random.seed(datetime.now())
import numpy as np
var('A,B,C,D')
\end{sagesilent}

%hhhe


%horizontal line

%ppps5

\begin{sagesilent}
A = randint(10,100)
B = randint(10,200)
C = randint(10,300)
D = B
P = line([[A,B],[C,D]])
Q = point([A,B])
S = point([C,D])
\end{sagesilent}

\sageplot{P+Q+S}

\begin{problem}{1}
Given $A(\sage{A},\,\sage{B})$ and $B(\sage{C},\,\sage{D})$,
$\mid \overline{AB}\mid $.
\end{problem}
%pppe
%hhhs

\documentclass[12pt]{article}

\usepackage[margin=1in]{geometry} 
\usepackage{amsmath,amsthm,amssymb}
\usepackage{float}
\usepackage[pdftex]{graphicx}
\usepackage{sagetex}
\usepackage{ifthen}
\usepackage{sagetex}

 

\newcommand{\N}{\mathbb{N}}
\newcommand{\Z}{\mathbb{Z}}

 

\newenvironment{theorem}[2][Theorem]{\begin{trivlist}
\item[\hskip \labelsep {\bfseries #1}\hskip \labelsep {\bfseries #2.}]}{\end{trivlist}}

\newenvironment{lemma}[2][Lemma]{\begin{trivlist}
\item[\hskip \labelsep {\bfseries #1}\hskip \labelsep {\bfseries #2.}]}{\end{trivlist}}

\newenvironment{exercise}[2][Exercise]{\begin{trivlist}
\item[\hskip \labelsep {\bfseries #1}\hskip \labelsep {\bfseries #2.}]}{\end{trivlist}}

\newenvironment{problem}[2][Problem]{\begin{trivlist}
\item[\hskip \labelsep {\bfseries #1}\hskip \labelsep {\bfseries #2.}]}{\end{trivlist}}

\newenvironment{question}[2][Question]{\begin{trivlist}
\item[\hskip \labelsep {\bfseries #1}\hskip \labelsep {\bfseries #2.}]}{\end{trivlist}}

\newenvironment{corollary}[2][Corollary]{\begin{trivlist}
\item[\hskip \labelsep {\bfseries #1}\hskip \labelsep {\bfseries #2.}]}{\end{trivlist}}

 \newenvironment{example}[2][Example]{\begin{trivlist}
\item[\hskip \labelsep {\bfseries #1}\hskip \labelsep {\bfseries #2.}]}{\end{trivlist}}
















\begin{document}
\title{Homework 3}
\author{test}
\maketitle



\begin{sagesilent}
from random import randint
import random
from datetime import datetime
random.seed(datetime.now())
import numpy as np
var('A,B,C,D')
\end{sagesilent}

%hhhe


%horizontal line

%ppps5

\begin{sagesilent}
A = randint(10,100)
B = randint(10,200)
C = randint(10,300)
D = B
P = line([[A,B],[C,D]])
Q = point([A,B])
S = point([C,D])
\end{sagesilent}

\sageplot{P+Q+S}

\begin{problem}{1}
Given $A(\sage{A},\,\sage{B})$ and $B(\sage{C},\,\sage{D})$,
$\mid \overline{AB}\mid $.
\end{problem}
%pppe







% both horizontal and vertical

%ppps5

\begin{sagesilent}
var('m')
y = function('y')
m = randint(-5,5)
A = randint(10,100)
B = randint(10,200)
C = randint(10,300)
D = B
E = C
F = randint(10, 300)

P = line([[A,B],[C,D]])
Q = line([[C,D],[E,F]])
R = point([A,B])
S = point([C,D])
T = point([E,F])


LA = text('A', (A,B+.03*max(D,B,max(y(A),y(C)))))
LB = text('B', (C+.03*C, D))
LC = text('C', (1.03*E, F))

\end{sagesilent}

\sageplot{P+Q+R+S+T+LA+LB+LC}

\begin{problem}{5}
Given $A(\sage{A},\,\sage{B})$ and $B(\sage{C},\,\sage{D})$,
calculate the distance $\mid \overline{AB} \mid$.
\end{problem}

%pppe


%vertical line

%ppps5

\begin{sagesilent}
A = randint(10,100)
B = randint(10,200)
C = A
D = randint(10,50)
P = line([[A,B],[C,D]])
Q = point([A,B])
S = point([C,D])
\end{sagesilent}

\sageplot{P+Q+S}

\begin{problem}{2}
Given $A(\sage{A},\,\sage{B})$ and $B(\sage{C},\,\sage{D})$,
calculate the distance
$\mid \overline{AB}\mid $.
\end{problem}
%pppe


%vertical line

%ppps5

\begin{sagesilent}
A = randint(10,100)
B = randint(10,200)
C = A
D = randint(10,50)
P = line([[A,B],[C,D]])
Q = point([A,B])
S = point([C,D])
\end{sagesilent}

\sageplot{P+Q+S}

\begin{problem}{2}
Given $A(\sage{A},\,\sage{B})$ and $B(\sage{C},\,\sage{D})$,
calculate the distance
$\mid \overline{AB}\mid $.
\end{problem}
%pppe







% both horizontal and vertical

%ppps5

\begin{sagesilent}
var('m')
y = function('y')
m = randint(-5,5)
A = randint(10,100)
B = randint(10,200)
C = randint(10,300)
D = B
E = C
F = randint(10, 300)

P = line([[A,B],[C,D]])
Q = line([[C,D],[E,F]])
R = point([A,B])
S = point([C,D])
T = point([E,F])


LA = text('A', (A,B+.03*max(D,B,max(y(A),y(C)))))
LB = text('B', (C+.03*C, D))
LC = text('C', (1.03*E, F))

\end{sagesilent}

\sageplot{P+Q+R+S+T+LA+LB+LC}

\begin{problem}{5}
Given $A(\sage{A},\,\sage{B})$ and $B(\sage{C},\,\sage{D})$,
calculate the distance $\mid \overline{AB} \mid$.
\end{problem}

%pppe
%hhhs

\documentclass[12pt]{article}

\usepackage[margin=1in]{geometry} 
\usepackage{amsmath,amsthm,amssymb}
\usepackage{float}
\usepackage[pdftex]{graphicx}
\usepackage{sagetex}
\usepackage{ifthen}
\usepackage{sagetex}

 

\newcommand{\N}{\mathbb{N}}
\newcommand{\Z}{\mathbb{Z}}

 

\newenvironment{theorem}[2][Theorem]{\begin{trivlist}
\item[\hskip \labelsep {\bfseries #1}\hskip \labelsep {\bfseries #2.}]}{\end{trivlist}}

\newenvironment{lemma}[2][Lemma]{\begin{trivlist}
\item[\hskip \labelsep {\bfseries #1}\hskip \labelsep {\bfseries #2.}]}{\end{trivlist}}

\newenvironment{exercise}[2][Exercise]{\begin{trivlist}
\item[\hskip \labelsep {\bfseries #1}\hskip \labelsep {\bfseries #2.}]}{\end{trivlist}}

\newenvironment{problem}[2][Problem]{\begin{trivlist}
\item[\hskip \labelsep {\bfseries #1}\hskip \labelsep {\bfseries #2.}]}{\end{trivlist}}

\newenvironment{question}[2][Question]{\begin{trivlist}
\item[\hskip \labelsep {\bfseries #1}\hskip \labelsep {\bfseries #2.}]}{\end{trivlist}}

\newenvironment{corollary}[2][Corollary]{\begin{trivlist}
\item[\hskip \labelsep {\bfseries #1}\hskip \labelsep {\bfseries #2.}]}{\end{trivlist}}

 \newenvironment{example}[2][Example]{\begin{trivlist}
\item[\hskip \labelsep {\bfseries #1}\hskip \labelsep {\bfseries #2.}]}{\end{trivlist}}
















\begin{document}
\title{Homework 3}
\author{test}
\maketitle



\begin{sagesilent}
from random import randint
import random
from datetime import datetime
random.seed(datetime.now())
import numpy as np
var('A,B,C,D')
\end{sagesilent}

%hhhe


%horizontal line

%ppps5

\begin{sagesilent}
A = randint(10,100)
B = randint(10,200)
C = randint(10,300)
D = B
P = line([[A,B],[C,D]])
Q = point([A,B])
S = point([C,D])
\end{sagesilent}

\sageplot{P+Q+S}

\begin{problem}{1}
Given $A(\sage{A},\,\sage{B})$ and $B(\sage{C},\,\sage{D})$,
$\mid \overline{AB}\mid $.
\end{problem}
%pppe


%vertical line

%ppps5

\begin{sagesilent}
A = randint(10,100)
B = randint(10,200)
C = A
D = randint(10,50)
P = line([[A,B],[C,D]])
Q = point([A,B])
S = point([C,D])
\end{sagesilent}

\sageplot{P+Q+S}

\begin{problem}{2}
Given $A(\sage{A},\,\sage{B})$ and $B(\sage{C},\,\sage{D})$,
calculate the distance
$\mid \overline{AB}\mid $.
\end{problem}
%pppe


%vertical line

%ppps5

\begin{sagesilent}
A = randint(10,100)
B = randint(10,200)
C = A
D = randint(10,50)
P = line([[A,B],[C,D]])
Q = point([A,B])
S = point([C,D])
\end{sagesilent}

\sageplot{P+Q+S}

\begin{problem}{2}
Given $A(\sage{A},\,\sage{B})$ and $B(\sage{C},\,\sage{D})$,
calculate the distance
$\mid \overline{AB}\mid $.
\end{problem}
%pppe
\end{document}