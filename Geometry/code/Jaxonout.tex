\documentclass[12pt]{article}

			\usepackage[margin=1in]{geometry} 
			\usepackage{amsmath,amsthm,amssymb}
			\usepackage{float}
			\usepackage[pdftex]{graphicx}

			 

			\newcommand{\N}{\mathbb{N}}
			\newcommand{\Z}{\mathbb{Z}}

			 

			\newenvironment{theorem}[2][Theorem]{\begin{trivlist}
			\item[\hskip \labelsep {\bfseries #1}\hskip \labelsep {\bfseries #2.}]}{\end{trivlist}}

			\newenvironment{lemma}[2][Lemma]{\begin{trivlist}
			\item[\hskip \labelsep {\bfseries #1}\hskip \labelsep {\bfseries #2.}]}{\end{trivlist}}

			\newenvironment{exercise}[2][Exercise]{\begin{trivlist}
			\item[\hskip \labelsep {\bfseries #1}\hskip \labelsep {\bfseries #2.}]}{\end{trivlist}}

			\newenvironment{problem}[2][Problem]{\begin{trivlist}
			\item[\hskip \labelsep {\bfseries #1}\hskip \labelsep {\bfseries #2.}]}{\end{trivlist}}

			\newenvironment{question}[2][Question]{\begin{trivlist}
			\item[\hskip \labelsep {\bfseries #1}\hskip \labelsep {\bfseries #2.}]}{\end{trivlist}}

			\newenvironment{corollary}[2][Corollary]{\begin{trivlist}
			\item[\hskip \labelsep {\bfseries #1}\hskip \labelsep {\bfseries #2.}]}{\end{trivlist}}

			 

			\begin{document}



			\title{HW 2}
			\author{Jaxon}
			\date{}
			\maketitle
I've completely randomized the following problems so you won't be able to cheat. Recall how to 
graph on the real number line. If $x$ is taken
to be your variable, you will shade in the regions for which the statement is true. For example, if you're
trying to plot $x < 10$, you will shade in everything to the left of $x = 10$ (because everything to the 
left of $x = 10$ is where $x < 10$) and leave unshaded everything to the right of $x = 10$ because that's where
$x > 10$.

Since the problems are randomized, you may have a contradiction such as $x < 10$ and $x > 11$. If that's the case,
don't plot anything and write $\phi$ down.

Also there's a challenge problem at the end. This one is worth 2 homework passes.

			\begin{problem}{0} 
On the real number line, plot $ x \leq -6 \;\rm{ or }\; x \geq 2$ and describe the geometric object.
 \end{problem}\begin{problem}{1} 
On the real number line, plot $ x \leq -1 \;\rm{ or }\; x \geq 6$ and describe the geometric object.
 \end{problem}\begin{problem}{2} 
On the real number line, plot $\mid x\mid \geq 10$ and describe the geometric object.
 \end{problem}\begin{problem}{3} 
On the real number line, plot $\mid x\mid \leq -10$ and describe the geometric object.
 \end{problem}\begin{problem}{4} 
On the real number line, plot $x \leq 6$ and describe the geometric object.
 \end{problem}\begin{problem}{5} 
On the real number line, plot $ x \leq -7 \;\rm{ or }\; x \geq -7$ and describe the geometric object.
 \end{problem}\begin{problem}{6} 
On the real number line, plot $\mid x\mid \leq 3$ and describe the geometric object.
 \end{problem}\begin{problem}{7} 
On the real number line, plot $\mid x\mid \geq -4$ and describe the geometric object.
 \end{problem}\begin{problem}{8} 
On the real number line, plot $ x \leq 3 \;\rm{ and }\; x \geq -4$ and describe the geometric object.
 \end{problem}\begin{problem}{9} 
On the real number line, plot $ x \leq 6 \;\rm{ or }\; x \geq 7$ and describe the geometric object.
 \end{problem}\begin{problem}{10} 
On the real number line, plot $x \geq -6$ and describe the geometric object.
 \end{problem}\begin{problem}{11} 
On the real number line, plot $ x \leq -10 \;\rm{ and }\; x \geq 7$ and describe the geometric object.
 \end{problem}\begin{problem}{Challenge} 

	Recall that a \textit{postulate} is a statement that we \textit{define} as being true. It cannot
	be logically deduced from other postulates like a \textit{theorem} can.
	The Segment Addition Postulate and Ruler Postulate seem redundant. Are they redundant? Why or why not?
	
 \end{problem}\end{document}