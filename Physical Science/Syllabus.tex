\documentclass[11pt,article,oneside]{memoir}
\usepackage[left=2cm, right=2cm, top=2cm]{geometry}

\setsecheadstyle{\Large\scshape}
\setsubsecheadstyle{\large\scshape\hspace{5mm}}

\begin{document}

\title{\LARGE Physical Science}
\author{\Large Mr. Braunberger \newline \\ \footnotesize\texttt{\noindent jake.braunberger@gmail.com}}
\date{\hspace*{2.3cm}Fall 2018 \newline \noindent \hspace*{-1.4cm} 10:11am--11:01am}

\setlength{\epigraphwidth}{.8\textwidth}

\maketitle

\epigraph{I think it is very important--at least it was to me--that if you are going to teach people to make observations, you should show that something wonderful can come from them.}{\textit{--Richard Feynman}}




\section*{Course Description}
Physical science is a branch of natural science that studies non-living systems, in contrast to life science. It in turn has many branches, each referred to as a ``physical science", together called the ``physical sciences".\footnote{Wikipedia} Upon successful completion
 of this course, the student will, among other things, have
developed critical thinking skills and have an appreciation for science.


\section*{Procedures, Materials, and Expectations}
\subsection*{Materials}
	Everyday, the student is expected to bring to class their course binder, textbook, a pencil or pen, and a notebook. 
\subsection*{Expectations}
	Science is a subject that builds upon itself and therefore requires study on a regular basis.
	Issued assignments are to be completed on time. Late work will be accepted; however,
late assignments will be automatically docked 30\%. Assignments may be completed in pen or pencil, as long as the assignment is neat.

The course will model scientific/mathematical thought: errors are to be expected, discovered, and remedied. Any incorrect answers
on homework will be awarded 50\% credit if recompleted correctly within a week of receiving the homework.
\subsection*{Grading}
	Tests and quizzes will be worth 15\% of the grade. The final will be worth 20\% of the grade. Homework will be worth 65\% of the grade.




\section*{The Book}
The book used for the course will be \textit{blahblah} by [Author]. Lecture notes to supplement class will be disseminated
as necessary.


\section*{Course Outline}
\subsection*{Introduction to Physical Science}
\begin{enumerate}
\item Desserts not deserts.
\item Flavors and toppings of physical science.
\end{enumerate}
\subsection*{Introduction to Physics}
\begin{enumerate}
\item Flavors of physics.
\item Compare and contrast the branches of physics.
\item GUT.
\item M(hmm)--theory.
\end{enumerate}
\subsection*{Classical Physics}
\begin{enumerate}
\item Directions: a double entendre.
\item Motion and momentum.
\item Forces and effects on motion.
\item Various types of forces.
\item Energy: potential and kinetic.
\end{enumerate}
\subsection*{Electricity and Magnetism}
\begin{enumerate}
\item Plumbing.
\item Light bulbs.
\item Computers.
\item Protons and electrons: an introduction.
\item Van de Graaf.
\item Magnets and magnetic fields.
\end{enumerate}
\subsection*{Quantum Mechanics}
\begin{enumerate}
\item Atoms and molecules.
\item Various models of atom. Electron cloud. Terminology.
\item Particle identity.
\item Superposition.
\item Exchange symmetry.
\item Pauli exclusion principle.
\item Phonons and photons are bosons: superconductivity.
\item The no clone theorem and perfect encryption.
\item Quantum computing.
\end{enumerate}
\subsection*{Astronomy not Gastronomy}
\begin{enumerate}
\item Turtles all the way down.
\item Location, location, location.
\item Perspective and more perspective.
\item Planetarium.
\end{enumerate}
\subsection*{Special Relativity}
\begin{enumerate}
\item Maxwell's equations.
\item The ultimate speed limit: $c + 100 = c$
\item Length contraction and time dilation.
\item Time machines.
\end{enumerate}
\subsection*{Gravity}
\begin{enumerate}
\item Space-time.
\item Elevators: the equivalence principle.
\item Matter space-time interaction: golfing on a trampoline.
\end{enumerate}


\end{document}