\documentclass[12pt]{article}

\usepackage[margin=1in]{geometry} 
\usepackage{amsmath,amsthm,amssymb}
\usepackage{float}
\usepackage[pdftex]{graphicx}
\usepackage{sagetex}

 

\newcommand{\N}{\mathbb{N}}
\newcommand{\Z}{\mathbb{Z}}

 

\newenvironment{theorem}[2][Theorem]{\begin{trivlist}
\item[\hskip \labelsep {\bfseries #1}\hskip \labelsep {\bfseries #2.}]}{\end{trivlist}}

\newenvironment{lemma}[2][Lemma]{\begin{trivlist}
\item[\hskip \labelsep {\bfseries #1}\hskip \labelsep {\bfseries #2.}]}{\end{trivlist}}

\newenvironment{exercise}[2][Exercise]{\begin{trivlist}
\item[\hskip \labelsep {\bfseries #1}\hskip \labelsep {\bfseries #2.}]}{\end{trivlist}}

\newenvironment{problem}[2][Problem]{\begin{trivlist}
\item[\hskip \labelsep {\bfseries #1}\hskip \labelsep {\bfseries #2.}]}{\end{trivlist}}

\newenvironment{question}[2][Question]{\begin{trivlist}
\item[\hskip \labelsep {\bfseries #1}\hskip \labelsep {\bfseries #2.}]}{\end{trivlist}}

\newenvironment{corollary}[2][Corollary]{\begin{trivlist}
\item[\hskip \labelsep {\bfseries #1}\hskip \labelsep {\bfseries #2.}]}{\end{trivlist}}

 \newenvironment{example}[2][Example]{\begin{trivlist}
\item[\hskip \labelsep {\bfseries #1}\hskip \labelsep {\bfseries #2.}]}{\end{trivlist}}

\begin{document}
\title{Homework 1}
\author{Hannah}
\maketitle
For a review of how to convert units, check out 
https://youtu.be/XKCZn5MLKvk

\begin{example}{1}
Suppose there are 200 apples in one bushel. Convert 2 apples to bushels.

Note that
\begin{equation}
\label{eq:conv}
	200 \; \rm{apples} = 1 \; \rm{bushel}
\end{equation}

This will be used to determine our conversion factor later.

Start by writing the quantity and units we want to convert on the left and the units we want converted
into on the right like this:
\[
2 \; \rm{apples} \cdot \frac{\rm{conversion \;factor}}{\rm{conversion \;factor}} = \rm{?} \; \rm{bushels}
\]

To determine the conversion factor, look at Eq. \ref{eq:conv} above. The units `apples' have to go in the
bottom of the fraction since that's what we're converting \textit{from}.

So we have
\[
	2 \; \rm{apples} \cdot \frac{1 \; \rm{bushel}}{200 \; \rm{apples}} = \rm{?} \; \rm{bushels}
\]
Performing the multiplication, we find $2$ apples $= 0.01$ bushels.

\end{example}

\begin{sagesilent}
from random import randint
var('A,B,C,D')
A = randint(10,100)
B = randint(10,200)
C = randint(10,300)
D = randint(10,50)
\end{sagesilent}

\begin{problem}{1}
Suppose there are $\sage{A}$ crows in a murder. Convert 2 crows to murders.
\end{problem}

\begin{problem}{2}
Suppose there are $\sage{B}$ Sickles in a Galleon. Convert 20 Sickles to Galleons.
\end{problem}

\begin{problem}{3}
Suppose there are $\sage{C}$ pounds of lumber in a chord. Convert 3 chords to pounds of lumber.
\end{problem}

\begin{problem}{4}
Suppose there are $\sage{D}$ parrots in a pandemonium. Convert 4 pandemoniums to parrots.
\end{problem}




\end{document}